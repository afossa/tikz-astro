% TikZ drawings of astrodynamics concepts
% Copyright (C) 2023  Alberto Fossa'

% This program is free software: you can redistribute it and/or modify
% it under the terms of the GNU Affero General Public License as published by
% the Free Software Foundation, either version 3 of the License, or
% (at your option) any later version.

% This program is distributed in the hope that it will be useful,
% but WITHOUT ANY WARRANTY; without even the implied warranty of
% MERCHANTABILITY or FITNESS FOR A PARTICULAR PURPOSE.  See the
% GNU Affero General Public License for more details.

% You should have received a copy of the GNU Affero General Public License
% along with this program.  If not, see <https://www.gnu.org/licenses/>.

\documentclass[multi]{standalone}

\usepackage{physics}
\usepackage[svgnames]{xcolor}
\usepackage{tikz,tikz-3dplot}
\usetikzlibrary{arrows.meta}

\begin{document}

% definition of Classical Orbital Elements
\tdplotsetmaincoords{70}{110}
\begin{tikzpicture}[tdplot_main_coords,scale=5]

    \tikzset{>=stealth}
    \pgfmathsetmacro{\r}{.8}
    \pgfmathsetmacro{\W}{45} % right ascension of ascending node
    \pgfmathsetmacro{\i}{30} % inclination
    \pgfmathsetmacro{\w}{80} % argument of periapsis
    \pgfmathsetmacro{\v}{35} % true anomaly

    \pgfmathsetmacro{\c}{\W-90} % first rotation about z
    \pgfmathsetmacro{\b}{\i}    % second rotation about y'
    \pgfmathsetmacro{\a}{\w+90} % third rotation about z''
    \pgfmathsetmacro{\u}{\a+\v} % third rotation about z''

    \coordinate (O) at (0,0,0) node[below] {$O$};
    \draw[->,thick] (O) -- (2,0,0) node[below] {$X$};
    \draw[->,thick] (O) -- (0,1.2,0) node[right] {$Y$};
    \draw[->,thick] (O) -- (0,0,1) node[right] {$Z$};

    \tdplotdrawarc[dashed]{(O)}{\r}{0}{360}{}{}
    \tdplotdrawarc[->,thick,Red]{(O)}{.4*\r}{0}{\W}{below}{$\Omega$}
    \node[text width=5em] at (0,-1.15*\r,0) {Reference plane};
    \node at (1.7,.4,0) {Orbit plane};

    \tdplotsetrotatedcoords{\W}{0}{0}
    \draw[tdplot_rotated_coords,->,thick] (-1,0,0) -- (1.5,0,0) node[below right] {Line of nodes};

    \tdplotsetrotatedcoords{\c}{\b}{0}
    \tdplotdrawarc[tdplot_rotated_coords]{(O)}{\r}{0}{360}{}{}
    \begin{scope}[tdplot_rotated_coords]
        \draw[->,thick] (O) -- (0,0,1) node[left,yshift=-3] {$\vb*{h}$};
        \tdplotdrawarc[->,thick,Blue]{(O)}{.4*\r}{90}{90+\w}{right}{$\omega$}
        \tdplotdrawarc[->,thick,Green]{(O)}{.4*\r}{90+\w}{90+\w+\v}{above right,yshift=-3}{$\nu$}
    \end{scope}

    \tdplotsetrotatedcoords{\c}{\b}{\a}
    \begin{scope}[tdplot_rotated_coords]
        \draw[->,thick] (O) -- (1,0,0) node[below right,xshift=-3] {$\vb*{e}$};
        \coordinate (R) at (\r,0,0);
    \end{scope}

    \tdplotsetrotatedcoords{\c}{\b}{\u}
    \begin{scope}[tdplot_rotated_coords]
        \draw[->,thick] (O) -- (\r,0,0) node[below right,xshift=-3] {$\vb*{r}$};
        \coordinate (P) at (\r,0,0);
    \end{scope}

    \coordinate (N) at (\W:\r);
    \tdplotsetrotatedcoords{\c-90}{90}{0}
    \begin{scope}[tdplot_rotated_coords]
        \tdplotdrawarc[->,thick,DarkViolet]{(O)}{.75*\r}{180}{180-\i}{above right,yshift=2}{$i$}
        \tdplotdrawarc[->,thick,DarkViolet]{(N)}{.25*\r}{-93}{-94-\i}{right,yshift=4}{$i$}
    \end{scope}

    \tdplotsetrotatedcoords{\c}{\b}{\u+90}
    \tdplotsetrotatedcoordsorigin{(P)}
    \begin{scope}[tdplot_rotated_coords]
        \draw [->,thick] (P) -- (.25,0,0) node [above] {$\vb*{v}$};
    \end{scope}

    \fill (P) circle (.05ex) node[above] {$P$};
    \fill (N) circle (.05ex) node[below right,xshift=6,yshift=6] {Ascending node};
    \fill (R) circle (.05ex) node[below right,xshift=2,yshift=2] {Perigee};

\end{tikzpicture}

\newpage

% rotations in TikZ
\tdplotsetmaincoords{70}{110}
\begin{tikzpicture}[tdplot_main_coords,scale=5]

    \tikzset{>=stealth}
    \pgfmathsetmacro{\r}{.5}
    \pgfmathsetmacro{\ph}{55}
    \pgfmathsetmacro{\th}{75}
    \pgfmathsetmacro{\ps}{45}

    \coordinate (O) at (0,0,0) node[below] {$O$};
    \draw[->,thick] (O) -- (1,0,0) node[below] {$X$};
    \draw[->,thick] (O) -- (0,1,0) node[right] {$Y$};
    \draw[->,thick] (O) -- (0,0,1) node[right] {$Z$};

    \tdplotdrawarc[->,thick,blue]{(O)}{\r}{0}{\ph}{below}{$\phi$}

    \tdplotsetrotatedcoords{\ph}{0}{0}
    \begin{scope}[tdplot_rotated_coords]
        \draw[->,thick,blue] (O) -- (1,0,0) node[below] {$x'$};
        \draw[->,thick,blue] (O) -- (0,1,0) node[right] {$y'$};
        \draw[->,thick,blue] (O) -- (0,0,1) node[left] {$z'$};
    \end{scope}

    \tdplotsetthetaplanecoords{\ph}
    \tdplotdrawarc[tdplot_rotated_coords,->,thick,red]{(O)}{\r}{90}{90+\th}{right}{$\theta$}

    \tdplotsetrotatedcoords{\ph}{\th}{0}
    \begin{scope}[tdplot_rotated_coords]
        \draw[->,thick,red] (O) -- (1,0,0) node[below] {$x''$};
        \draw[->,thick,red] (O) -- (0,1,0) node[above] {$y''$};
        \draw[->,thick,red] (O) -- (0,0,1) node[right] {$z''$};
        \tdplotdrawarc[->,thick,green]{(O)}{\r}{0}{\ps}{below}{$\psi$}
    \end{scope}

    \tdplotsetrotatedcoords{\ph}{\th}{\ps}
    \begin{scope}[tdplot_rotated_coords]
        \draw[->,thick,green] (O) -- (1,0,0) node[below] {$x'''$};
        \draw[->,thick,green] (O) -- (0,1,0) node[right] {$y'''$};
        \draw[->,thick,green] (O) -- (0,0,1) node[above] {$z'''$};
    \end{scope}

\end{tikzpicture}

\newpage

% % definition of Radial, along-track, cross-track frame
\tdplotsetmaincoords{70}{110}
\begin{tikzpicture}[tdplot_main_coords,scale=5]

    \tikzset{>=stealth}
    \pgfmathsetmacro{\a}{1}
    \pgfmathsetmacro{\e}{.5}
    \pgfmathsetmacro{\W}{30} % right ascension of ascending node
    \pgfmathsetmacro{\i}{30} % inclination
    \pgfmathsetmacro{\w}{-90} % argument of periapsis
    \pgfmathsetmacro{\v}{150} % true anomaly

    \pgfmathsetmacro{\b}{\a*sqrt(1-\e*\e)} % semi-minor axis
    \pgfmathsetmacro{\r}{\a*(1-\e*\e)/(1+\e*cos(\v))} % radius
    \pgfmathsetmacro{\x}{\r*cos(\v)} % x coordinate
    \pgfmathsetmacro{\y}{\r*sin(\v)} % y coordinate
    \pgfmathsetmacro{\vx}{-sin(\v)} % x velocity
    \pgfmathsetmacro{\vy}{(\e+cos(\v))} % y velocity

    \pgfmathsetmacro{\c}{\W-90} % first rotation about z
    \pgfmathsetmacro{\bt}{\i}    % second rotation about y'
    \pgfmathsetmacro{\al}{\w+90} % third rotation about z''
    \pgfmathsetmacro{\u}{\al+\v} % third rotation about z''

    \coordinate (O) at (0,0,0) node[below] {$O$};
    \draw[->,thick] (O) -- (1.5,0,0) node[below] {$X$};
    \draw[->,thick] (O) -- (0,1.5,0) node[right] {$Y$};
    \draw[->,thick] (O) -- (0,0,1) node[right] {$Z$};

    \tdplotsetrotatedcoords{\c}{\bt}{\al}
    \begin{scope}[tdplot_rotated_coords]
        \draw (-\a*\e,0,0) ellipse [x radius=\a,y radius=\b];
        \draw[->,thick] (O) -- (\x,\y,0) node[below,yshift=-3] {$\vb*{r}$};
        \draw[->,thick] (O) -- (0,0,1) node[left,yshift=-3] {$\vb*{h}$};
        \coordinate (P) at (\x,\y,0);
    \end{scope}

    \tdplotsetrotatedcoordsorigin{(P)}
    \begin{scope}[tdplot_rotated_coords,scale=.75]
        \draw[->,thick] (P) -- (\vx,\vy,0) node[below,xshift=4] {$\vb*{v}$};
    \end{scope}

    \tdplotsetrotatedcoords{\c}{\bt}{\u}
    \tdplotsetrotatedcoordsorigin{(P)}
    \begin{scope}[tdplot_rotated_coords,scale=.25]
        \draw[->,thick,Red] (P) -- (1,0,0) node[below,xshift=4] {$\hat{\vb*{e}}_r$};
        \draw[->,thick,Red] (P) -- (0,1,0) node[above] {$\hat{\vb*{e}}_s$};
        \draw[->,thick,Red] (P) -- (0,0,1) node[below left,xshift=6] {$\hat{\vb*{e}}_w$};
    \end{scope}

    \fill (P) circle (.05ex);
    \node at (1.1,0.5,0) {Osculating orbit};

\end{tikzpicture}

\newpage

% definition of the CR3BP synodic frame
\tdplotsetmaincoords{60}{-15}
\begin{tikzpicture}[tdplot_main_coords,scale=5]

    \tikzset{>={Stealth}}
    \pgfmathsetmacro{\ph}{30}
    \pgfmathsetmacro{\lwi}{.5mm}
    \pgfmathsetmacro{\lwr}{.25mm}
    \pgfmathsetmacro{\rw}{0.1}

    \coordinate (O) at (0,0,0);
    \draw[->,line width=\lwi] (O) -- (1,0,0) node[below] {$X$};
    \draw[->,line width=\lwi] (O) -- (0,1,0) node[below, xshift=-5, yshift=-5] {$Y$};
    \draw[->,line width=\lwi] (O) -- (0,0,1) node[right] {$Z$};

    \tdplotsetrotatedcoords{\ph}{0}{0}
    \begin{scope}[tdplot_rotated_coords]
        \draw[line width=\lwr] (O) -- (-.5,0,0);
        \draw[->,line width=\lwr] (O) -- (1,0,0) node[below] {$x$};
        \draw[->,line width=\lwr] (O) -- (0,1,0) node[below, yshift=-2] {$y$};
        \draw[->,line width=\lwr] (O) -- (0,0,1) node[left] {$z$};
        \coordinate (P1) at (-.25,0,0);
        \coordinate (P2) at (.75,0,0);
        \coordinate (P) at (.75,.25,.5);
        \draw[->,line width=\lwr] (P1) -- (P) node[midway, left, xshift=15, yshift=25] {$\vb*{r}_1$};
        \draw[->,line width=\lwr] (P2) -- (P) node[midway, right] {$\vb*{r}_2$};
        \draw[->,line width=\lwr] (O) -- (P) node[midway, right] {$\vb*{r}$};
    \end{scope}

    \coordinate (A) at (0,0,.85);
    \tdplotsetrotatedcoordsorigin{(A)}
    \begin{scope}[tdplot_rotated_coords]
        \draw[->,line width=\lwi,red] (-\rw,0,0) arc[start angle=90, end angle=390, radius=\rw] node[right, xshift=7] {$\vb*{\omega}$};
    \end{scope}

    \fill (O) circle (.1ex) node[below] {$O$};
    \fill (P1) circle (.1ex) node[below right] {$P_1$};
    \fill (P2) circle (.1ex) node[below right] {$P_2$};
    \fill (P) circle (.1ex) node[above right] {$P_3$};

\end{tikzpicture}

\end{document}